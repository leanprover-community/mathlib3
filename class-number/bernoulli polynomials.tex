\documentclass{beamer}
\usetheme{Madrid}

\usepackage{graphicx}
\graphicspath{ {C:} }
\usepackage{amsmath}
\usepackage{amssymb}
\usepackage{amsfonts}
\usepackage{hyperref}
\usepackage{makecell}
\usepackage{xcolor}
\usepackage{xspace}
\usepackage{listings}
\def\lstlanguagefiles{lstlean.tex}
\lstset{language=lean}

\definecolor{keywordcolor}{rgb}{0.7, 0.1, 0.1}   % red
\definecolor{commentcolor}{rgb}{0.4, 0.4, 0.4}   % grey
\definecolor{symbolcolor}{rgb}{0.4, 0.4, 0.4}    % grey
\definecolor{sortcolor}{rgb}{0.1, 0.5, 0.1}      % green

\renewcommand\UrlFont{\color{blue}\rmfamily}

%Information to be included in the title page:
\title{Bernoulli polynomials in Lean}
\author{Ashvni Narayanan}
\institute{Lean Summer Projects 2021}

\begin{document}

\frame{\titlepage}

\begin{frame}
\frametitle{My PhD project}
\begin{itemize}
  \item To formalize Iwasawa Theory Main Conjecture : \\
     $$ f_k((1 + p)^s - 1) = L_p(\omega^k, s) $$ \pause
  \item To formalize the p-adic L-functions : \\
    $$ L_p(-s, \chi) = \frac{-1}{1 - \chi(c)<c>^{s + 1}}
    \int_{(\mathbb{Z}/dp\mathbb{Z})^{\times} \times (1 + p \mathbb{Z}_p)} \chi \omega^{-1}(a) <a>^s dE_c $$ \pause
  \item To formalize the p-adic integral \pause
  \item To define the Bernoulli measure \pause
  \item To define the Bernoulli polynomial
\end{itemize}
\end{frame}

\begin{frame}
\frametitle{Bernoulli numbers}
The Bernoulli numbers are an important number theoretic object. They occur as special values
of the Reimann-zeta functions / p-adic L-functions. They are a generalization of
Bernoulli numbers. \pause

The Bernoulli numbers $B_n$ are generating functions given by :
$$\sum_{n = 0}^{\infty} B_n\frac{t^n}{n!}=\frac{t}{e^{t} - 1}$$ \pause

Note that several authors think of Bernoulli numbers $B_n'$ to be defined as :
$$\sum_{n = 0}^{\infty} B_n'\frac{t^n}{n!}=\frac{t}{1-e^{-t}}$$ \pause

The difference between these two is : $B_n' = (-1)^n B_n$, with $B_1 = \frac{-1}{2}$. \pause
\end{frame}

\begin{frame}[fragile]
\frametitle{Bernoulli numbers in Lean}
Using recursion, $B_n$ is defined in \lean{mathlib} as
$B_n := \sum_{k = 0}^{n - 1} {n \choose k} \frac{B_k}{n - k + 1}$ :
\begin{lstlisting}
  bernoulli' n = 1 - ∑ k : fin n, n.choose k / (n - k + 1) * bernoulli' k
\end{lstlisting} \pause
and $B_n'$ as
$B_n' := (-1)^n B_n $ :
\begin{lstlisting}
  def bernoulli (n : ℕ) : ℚ := (-1)^n * bernoulli' n
\end{lstlisting}
\end{frame}

\begin{frame}[fragile]
\frametitle{Bernoulli polynomials}
The Bernoulli polynomials denoted $B_n(X)$, a generalization of the Bernoulli numbers,
are generating functions :
$$ \sum_{n = 0}^{\infty} B_n(X) \frac{t^n}{n!} = \frac{t e^{tX}}{e^t - 1} $$ \pause

We now define the Bernoulli polynomials as
$ B_n(X) := \sum_{i = 0}^{n} {n \choose i} B_i X^{n - i} $ :
\begin{lstlisting}
  def bernoulli_poly (n : ℕ) : polynomial ℚ :=
  ∑ i in range (n + 1), polynomial.monomial (n - i) ((bernoulli i) * (choose n i))
\end{lstlisting}
\end{frame}

\begin{frame}[fragile]
\frametitle{Bernoulli polynomials in Lean}
The following properties of Bernoulli polynomials were proved :
\begin{enumerate}
  \item $ B_0(X) = X $ : \begin{lstlisting}
    lemma bernoulli_poly_zero : bernoulli_poly 0 = 1
  \end{lstlisting} \pause
  \item $ B_n(0) = B_n $ : \begin{lstlisting}
    lemma bernoulli_poly_eval_zero (n : ℕ) : (bernoulli_poly n).eval 0 = bernoulli n
  \end{lstlisting} \pause
  \item $ B_n(1) = B_n' $ : \begin{lstlisting}
    lemma bernoulli_poly_eval_one (n : ℕ) : (bernoulli_poly n).eval 1 = bernoulli' n
  \end{lstlisting}
\end{enumerate}
\end{frame}

\begin{frame}[fragile]
\frametitle{Bernoulli polynomials in Lean}
  \begin{enumerate}
  \item $\sum_{k = 0}^n {{n + 1} \choose k} B_k(X) = (n + 1) X^n $
  \begin{lstlisting}
    theorem sum_bernoulli_poly (n : ℕ) :
  ∑ k in range (n + 1), ((n + 1).choose k : ℚ) • bernoulli_poly k =
    polynomial.monomial n (n + 1 : ℚ)
  \end{lstlisting} \pause
  \item $ \bigg( \sum_{n = 0}^{\infty} B_n(t) \frac{X^n}{n!} \bigg) * (e^X - 1) = X e^{tX} $ :
  \begin{lstlisting}
    theorem exp_bernoulli_poly' (t : A) :
  mk (λ n, aeval t ((1 / n! : ℚ) • bernoulli_poly n)) * (exp A - 1) =
  X * rescale t (exp A)
  \end{lstlisting}
\end{enumerate}
\end{frame}

\begin{frame}
\frametitle{So..}
Come join us on Discord/Zulip! \pause
\begin{figure}
  \includegraphics[width = \linewidth]{dumbledore-quote.jpg}
\end{figure}
\end{frame}

\begin{frame}
Thank you!
\end{frame}
\end{document}
