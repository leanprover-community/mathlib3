\documentclass[a4paper]{article}
\usepackage[T1]{fontenc}
\usepackage[utf8]{inputenc}
\usepackage[english]{babel}

\usepackage[urw-garamond,expert,
uppercase=upright,greeklowercase=upright]{mathdesign}
\usepackage[osf,swashQ]{garamondx}

\usepackage{microtype}

\usepackage[textwidth=17cm, textheight=27cm]{geometry}
\usepackage{booktabs, xspace}
% \usepackage{amsmath,amsfonts,amssymb}

\newcommand{\lean}[1]{{\tt #1}}
\newcommand{\nv}{\textit{new\_name} }
\newcommand{\nom}{\textit{name} }
\newcommand{\expr}{\textit{expr} }
\newcommand{\fait}{\textit{fact} }
\newcommand{\hyp}{\textit{hyp}\xspace}


\begin{document}
\pagestyle{empty}
\begin{center}
 \large\textsc{Lean Cheatsheet}
\end{center}

In the following table,
\nom always refers to a name already known to Lean
while \nv refers to a new name provided by the user.
When one of these words appears twice in the same line,
the appearances do not designate the same name.
\expr designates an expression,
for example the name of an object in the context,
an arithmetic expression that is a function of such objects,
a hypothesis in the context,
or a lemma applied to any of these.

\begin{center}
\setlength\tabcolsep{1.4cm}
\def\arraystretch{1.5}
\begin{tabular}{@{}lll@{}}
  \toprule
  Logical symbol & Appears in goal & Appears in hypothesis \\
  \midrule
 $\forall$ (for all) & \lean{intro} \nv & \lean{apply} \expr or \lean{specialize} \nom \expr  \\
 $\exists$ (there exists) & \lean{use} \expr & \lean{cases} \expr \lean{with} \nv \nv \\
 $\to$ (implies) & \lean{intro} \nv & \lean{apply} \expr or \lean{specialize} \nom \expr \\
 $\leftrightarrow$ (if and only if) & \lean{split}  & \lean{rw} \expr or \lean{rw ←} \expr\\
 $\land$ (and) & \lean{split} & \lean{cases} \expr \lean{with} \nv \nv \\
 $\lor$ (or) & \lean{left} or \lean{right} & \lean{cases} \expr \lean{with} \nv \nv \\
 $\lnot$ (not) & \lean{intro} \nv & \lean{apply} \expr or \lean{specialize} \nom \expr  \\
  \bottomrule
\end{tabular}
\end{center}

\noindent
\emph{Note}:
Traditional paper-based practice uses $\Rightarrow$ for implication,
uses $\iff$ for equivalence,
and does not use a notation for ``and'', ``or'' and ``not''.

\medskip
\noindent
In the left-hand column of the following table,
the parts in brackets are optional.
The effect of these parts is also in brackets in the right-hand column.
It is almost always a matter of specifying that a manipulation,
which acts by default on the goal,
must be performed rather on a certain hypothesis named \hyp.

\begin{center}
\setlength\tabcolsep{.5cm}
\def\arraystretch{1.6}
\begin{tabular}{@{}lp{10cm}@{}}
  \toprule
  Tactic & Effect \\
  \midrule
  \lean{exact} \expr & asserts that the goal can be satisfied by \expr \\
  \lean{have} \nv : \fait & introduces a name \nv asserting that \fait is provable \\
  \lean{unfold} \nom (\lean{at} \hyp) & unfold the definition of \nom in the goal
  (or in the hypothesis \hyp) \\
  \lean{change} \expr (\lean{at} \hyp) & transform the goal (or the hypothesis \hyp)
  into the expression \expr to which it is equivalent by definition \\
  \lean{rw} (\lean{←}) \expr (\lean{at} \hyp) & in the goal (or in the
  hypothesis \hyp), replace the left-hand side
  (or the right-hand side, if \lean{←} is present)
  of the equality or equivalence \expr by the other side.
  The expression to be replaced must appear explicitly,
  one may use \lean{unfold} or \lean{change} to ensure this. \\
  \lean{linarith} & prove the goal by a linear combination of hypotheses \\
  \lean{ring} & prove the goal by combining the axioms of a commutative (semi)ring \\
  \lean{library\_search} & search for a single existing lemma which closes the goal, also using local hypotheses. \\
  \lean{choose} \nv \nv \lean{using} \expr &
  given \expr : $\forall x, \exists y, P(x, y)$,
  use the axiom of choice to produce a function $x \mapsto y(x)$ satisfying
  $\forall x, P(x, y(x))$\\
  \lean{exfalso} & apply the rule \emph{ex falso quod libet} \\
  \lean{by\_contradiction} \nv & start a proof by contradiction,
  using \nv as name for the hypothesis that is the negation of the goal \\
  \lean{by\_cases} \nv : \expr & split the proof into two cases
  depending on whether \expr is true or false,
  using \nv as name for this hypothesis \\
  \lean{contrapose} & transform a goal of the form \expr $\to$ \expr
  into its contrapositive \\
  \lean{push\_neg} (\lean{at} \hyp) & push negations in the goal
  (or in the hypothesis \hyp) \\
  \bottomrule
\end{tabular}
\end{center}

\end{document}
